\documentclass[10pt,twocolumn,letterpaper]{article}

\usepackage{cvpr}
\usepackage{times}
\usepackage{epsfig}
\usepackage{graphicx}
\usepackage{amsmath}
\usepackage{amssymb}

% Include other packages here, before hyperref.

% If you comment hyperref and then uncomment it, you should delete
% egpaper.aux before re-running latex.  (Or just hit 'q' on the first latex
% run, let it finish, and you should be clear).
\usepackage[breaklinks=true,bookmarks=false]{hyperref}

\cvprfinalcopy % *** Uncomment this line for the final submission

\def\cvprPaperID{****} % *** Enter the CVPR Paper ID here
\def\httilde{\mbox{\tt\raisebox{-.5ex}{\symbol{126}}}}

% Pages are numbered in submission mode, and unnumbered in camera-ready
%\ifcvprfinal\pagestyle{empty}\fi
\setcounter{page}{4321}
\begin{document}

%%%%%%%%% TITLE
\title{Bills Classificator}

\author{Bryan Lucchetta\\
{\small University of Padua}\\
{\tt\small bryan.lucchetta@studenti.unipd.it}
% For a paper whose authors are all at the same institution,
% omit the following lines up until the closing ``}''.
% Additional authors and addresses can be added with ``\and'',
% just like the second author.
% To save space, use either the email address or home page, not both
\and
Luca Parolari\\
{\small University of Padua}\\
{\tt\small luca.parolari@studenti.unipd.it}
}

\maketitle
%\thispagestyle{empty}

%%%%%%%%% ABSTRACT
\begin{abstract}
   Our lives are fully immersed by bills of every type. Every month we receive an average of 4-5 bills to pay about electricity, gas, water, internet and garbage furnitures and services. The goal of this work is to implement a classifier that automatically classify the bills that we received in the past. Given a batch of camera taken pictures of bills the application's task is to classify every bill and then store it inside the appropriate folder. In this way the app could organize better our finances lives. We build a sophisticated pipeline to handle the camera taken pictures, for then passing to an OCR that extract all the text contained in it. Then a simple Naive Bayes classifier is used in order to perform the classification. This is only the first step but it can be improved with more and sopphisticated features like the extraction of consumption, bill amount and deadlines and many more.
\end{abstract}

%%%%%%%%% BODY TEXT
\section{Introduction}

The purpose of our work is to classify bills about gas, water, garbage, internet and ellectricity furnitures and services. There are different aspects to take care of when dealing with camera taken photos, OCRs and document classification. For these reasons we develop a strong pipeline to manage different problems that could occur when dealing with these objects. Another important aspect when dealing with machine learning problems is getting a lot of data (in our case the bills) and the dataset construction. We will investigate all of these aspect in this paper. We are pretty sattisfied with this work because we reached a TODO\% of correct classification considering different metrics like... TODO    


\begin{figure*}[h!]
	\caption{Our pipeline}
	\centering
	\includegraphics[width=1.0\textwidth]{images/pipeline.png}
\end{figure*}


\section{Related Work}
We couldn't have finded work that perform this particular task on the internet. However different aspect that we had to take care when develop this project have been taken into consideration by several works: primarily the problem of dewarping a camera taken image is essential for our application, in particular for pre-processing the image for the OCR and this problem was faced by [TODO: Improving Camera-based Document Analysis with Deep Learning] wich is based upon the experiments conducted by [TODO Recovering Homography from Camera Captured Documents using Convolutionjal Neural Networs]. The second task is to extact all the words contained in a bill: to face this task we relied on Tesseract [TODO: set the refence] wich is a popular open-source OCR developed by Goggle. The last major step is performing the classification wich could be performed by a Naive Bayes Classifier, TODO set references. 

\section{Dataset}

\section{Method}

\section{Experiments}

\subsection{Dewarping or not?}
Explain how the results are affected by prospective issues.

\subsection{Camera taken image quality}
Brightness and contrast enhancer 

\section{Conclusion}




\subsection{Language}

All manuscripts must be in English.

\subsection{Suggested Structure}

The following is a suggested structure for your report:

\begin{itemize}
	\item Introduction (10\%): describe the problem you are working on, why it's important, and an overview of your results.
	\item Related Work (10\%): discuss published work or similar apps that relates to your project. How is your approach similar or different from others?
	\item Dataset (15\%): describe the data you are working with for your project. What type of data is it? Where did it come from? How much data are you working with? Did you have to do any preprocessing, filtering, etc., and why?
	\item Method (30\%): discuss your approach for solving the problems that you set up in the introduction. Why is your approach the right thing to do? Did you consider alternative approaches? It may be helpful to include figures, diagrams, or tables to describe your method or compare it with others.
	\item Experiments (30\%): discuss the experiments that you performed. The exact experiments will vary depending on the project, but you might compare with prior work, perform an ablation study to determine the impact of various components of your system, experiment with different hyperparameters or architectural choices. You should include graphs, tables, or other figures to illustrate your experimental results.
	\item Conclusion (5\%): summarize your key results; what have you learned? Suggest ideas for future extensions.
\end{itemize}	

%------------------------------------------------------------------------
\section{Formatting your paper}

All text must be in a two-column format. The total allowable width of the
text area is $6\frac78$ inches (17.5 cm) wide by $8\frac78$ inches (22.54
cm) high. Columns are to be $3\frac14$ inches (8.25 cm) wide, with a
$\frac{5}{16}$ inch (0.8 cm) space between them. The main title (on the
first page) should begin 1.0 inch (2.54 cm) from the top edge of the
page. The second and following pages should begin 1.0 inch (2.54 cm) from
the top edge. On all pages, the bottom margin should be 1-1/8 inches (2.86
cm) from the bottom edge of the page for $8.5 \times 11$-inch paper; for A4
paper, approximately 1-5/8 inches (4.13 cm) from the bottom edge of the
page.

%-------------------------------------------------------------------------
\subsection{Margins and page numbering}

All printed material, including text, illustrations, and charts, must be kept
within a print area 6-7/8 inches (17.5 cm) wide by 8-7/8 inches (22.54 cm)
high.
Page numbers should be in footer with page numbers, centered and .75
inches from the bottom of the page and make it start at the correct page
number rather than the 4321 in the example.  To do this fine the line (around
line 23)
\begin{verbatim}
%\ifcvprfinal\pagestyle{empty}\fi
\setcounter{page}{4321}
\end{verbatim}
where the number 4321 is your assigned starting page.

Make sure the first page is numbered by commenting out the first page being
empty on line 46
\begin{verbatim}
%\thispagestyle{empty}
\end{verbatim}


%-------------------------------------------------------------------------
\subsection{Type-style and fonts}

Wherever Times is specified, Times Roman may also be used. If neither is
available on your word processor, please use the font closest in
appearance to Times to which you have access.

MAIN TITLE. Center the title 1-3/8 inches (3.49 cm) from the top edge of
the first page. The title should be in Times 14-point, boldface type.
Capitalize the first letter of nouns, pronouns, verbs, adjectives, and
adverbs; do not capitalize articles, coordinate conjunctions, or
prepositions (unless the title begins with such a word). Leave two blank
lines after the title.

AUTHOR NAME(s) and AFFILIATION(s) are to be centered beneath the title
and printed in Times 12-point, non-boldface type. This information is to
be followed by two blank lines.

The ABSTRACT and MAIN TEXT are to be in a two-column format.

MAIN TEXT. Type main text in 10-point Times, single-spaced. Do NOT use
double-spacing. All paragraphs should be indented 1 pica (approx. 1/6
inch or 0.422 cm). Make sure your text is fully justified---that is,
flush left and flush right. Please do not place any additional blank
lines between paragraphs.

Figure and table captions should be 9-point Roman type as in
Table~\ref{mytable}. Short captions should be centred.

\noindent Callouts should be 9-point Helvetica, non-boldface type.
Initially capitalize only the first word of section titles and first-,
second-, and third-order headings.

FIRST-ORDER HEADINGS. (For example, {\large \bf 1. Introduction})
should be Times 12-point boldface, initially capitalized, flush left,
with one blank line before, and one blank line after.

SECOND-ORDER HEADINGS. (For example, { \bf 1.1. Database elements})
should be Times 11-point boldface, initially capitalized, flush left,
with one blank line before, and one after. If you require a third-order
heading (we discourage it), use 10-point Times, boldface, initially
capitalized, flush left, preceded by one blank line, followed by a period
and your text on the same line.

%-------------------------------------------------------------------------
\subsection{Footnotes}

Please use footnotes\footnote {This is what a footnote looks like.  It
often distracts the reader from the main flow of the argument.} sparingly.
Indeed, try to avoid footnotes altogether and include necessary peripheral
observations in
the text (within parentheses, if you prefer, as in this sentence).  If you
wish to use a footnote, place it at the bottom of the column on the page on
which it is referenced. Use Times 8-point type, single-spaced.


%-------------------------------------------------------------------------
\subsection{References}

List and number all bibliographical references in 9-point Times,
single-spaced, at the end of your paper. When referenced in the text,
enclose the citation number in square brackets, for
example~\cite{Authors14}.  Where appropriate, include the name(s) of
editors of referenced books.

\begin{table}
\begin{center}
\begin{tabular}{|l|c|}
\hline
Method & Frobnability \\
\hline\hline
Theirs & Frumpy \\
Yours & Frobbly \\
Ours & Makes one's heart Frob\\
\hline
\end{tabular}
\end{center}
\caption{Results. Ours is better.}
\label{mytable}
\end{table}

%-------------------------------------------------------------------------
\subsection{Illustrations, graphs, and photographs}

All graphics should be centered.  Please ensure that any point you wish to
make is resolvable in a printed copy of the paper.  Resize fonts in figures
to match the font in the body text, and choose line widths which render
effectively in print.  Many readers (and reviewers), even of an electronic
copy, will choose to print your paper in order to read it.  You cannot
insist that they do otherwise, and therefore must not assume that they can
zoom in to see tiny details on a graphic.

When placing figures in \LaTeX, it's almost always best to use
\verb+\includegraphics+, and to specify the  figure width as a multiple of
the line width as in the example below
{\small\begin{verbatim}
   \usepackage[dvips]{graphicx} ...
   \includegraphics[width=0.8\linewidth]
                   {myfile.eps}
\end{verbatim}
}


{\small
\bibliographystyle{ieee_fullname}
\bibliography{egbib}
}

\end{document}
